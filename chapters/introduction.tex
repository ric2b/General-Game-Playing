%!TEX root = ../dissertation.tex

\chapter{Introduction}
\label{chapter:introduction}

Game playing has always been a fundamental part of Artificial Intelligence (AI) research, as it can be used to test strategy in a straightforward way. Different games can be created to test specific features or properties that are of research interest, such as opponent modeling.

Throughout the history of AI research there was always a big focus on the ability of playing specific games, like chess, well. This focus led to systems like Deep Blue (the computer system that defeated Garry Kasparov at chess in the 90’s) that, while very advanced on the games they are designed for, delegate all the interesting analysis to the system designers. These systems are also completely useless for all games other than the one they were designed for (even if the difference is very small).

This over-specialization limits the usefulness of these systems. It is then of our interest to also be able to design more general systems, which could be closer to one of the most powerful features of human intelligence, adaptability:

\chapquote{``A human being should be able to (...) design a building, write a sonnet, balance accounts, build a wall, set a bone, comfort the dying, take orders, give orders, cooperate, act alone, solve equations, analyze a problem, pitch manure, program a computer, fight efficiently, die gallantly. \\ Specialization is for insects.''}{Robert A. Heinlein}{Time Enough for Love}

\gls{GGP} aims to develop general game playing systems, systems that can play any game when given the rules, acting as a stepping stone for General Intelligence research. Artificial General Intelligence has been a topic for science-fiction stories and, if possible, can be the biggest revolution in human history, allowing for an unprecedented ability of problem-solving.p
 

% A demonstration of how to use acronyms and glossary:

% A \gls{MSc} entry.

% Second use: \gls{IST}.

% Plurals: \glspl{MSc}.

% A citation example \cite{nobody}
