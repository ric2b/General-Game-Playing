\chapter{State of the Art}
\label{chapter:state_of_the_art}

\section{Competition as a benchmark}
The General Game Playing annual competition has been, since its creation in 2005 in Stanford University, the way to know which methods are the state of the art. 
While in the early years of the competition there was a bigger focus on intelligent heuristics, Monte Carlo Tree Search (MCTS) has dominated the competition ever since, since it’s domain independent (general), inherently parallel and has shown better performance in most of the tested games. One of the most important features of MCTS is that the process of building the tree can be paused when necessary (for example when a turn ends) and continued at a later time. This allows a player to continue the simulations throughout the whole game, without restarting after each turn.
The variant of MCTS commonly used in GGP is called Upper Confidence Bounds Applied for Trees (UCT), which provides a simple method to balance tree exploration (search new branches) and exploitation (search deeper in the known branches). 

There are other, smaller, \gls{GGP} competitions that usually have the same results in terms of what techniques tend to perform best.


\subsection{FluxPlayer}
The winner of the second GGP competition, in 2006, this player used fuzzy logic to determine how close to terminal a certain state is.

Fuzzy logic is a form of logic that allows multiple different values beyond True or False, and also allows overlap between these values. For example: water can be considered cold, warm or hot, but also warm and hot at the same time, since for some temperatures it is not clear which one is the correct one. Fuzzy logic also allows each of these states to have varying degrees of certainty: the water can be 80\% warm and 10\% cold, allowing conditions like if water is very warm: add some cold water, where the very keyword is also part of the fuzzy logic implementation.

The system also used a novel heuristic search that could be computed from the specifications of the game.

\subsection{Cadia Player}
Winner in 2007, started the reign of MCTS players in GGP.

\subsection{Sancho}
Champion in 2014, info here, need to research it: \url{http://sanchoggp.blogspot.pt/2014/05/what-is-sancho.html}

MCTS player with a few notable changes:
- propositional-network-based state machine (based on Petri Networks)
- general graph instead of MCTS tree
- some heuristics determined at game setup time

\subsection{Galvanize}
This years champion, MCTS based, research it:
\url{https://bitbucket.org/rxe/galvanise_v2}

\subsection{comparison}
research the top players from here:
\url{http://www.ggp.org/view/tiltyard/players/}
and here:
\url{https://docs.google.com/spreadsheets/d/12SWEXYAmCCGCm5jI-e0t1HGlMFXXeKSgyuU1boTioJ4/edit#gid=0&vpid=A1}
and make a comparison table.
If possible, run some of them and do my own comparisons.




