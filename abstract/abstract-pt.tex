%!TEX root = ../dissertation.tex

\begin{otherlanguage}{portuguese}
\begin{abstract}
\abstractPortuguesePageNumber
A investigação em Inteligência Artificial tem criado muitos sistemas com desempenho excepcional para as funções específicas para que foram desenhados, mas que são inúteis para qualquer outra função. Por exemplo, embora o Deep Blue tenha conseguido derrotar Garry Kasparov, o campeão mundial de xadrez, era completamente incapaz de jogar damas ou mesmo o tic-tac-toe. Para além disso, com estes sistemas, toda a análise interessante é feita previamente e é feita por quem desenvolve o sistema.
General Game Playing (GGP) tem como objectivo a criação de sistemas capazes de jogar qualquer jogo sem assistência humana, recebendo apenas as regras do jogo.
Sistemas deste tipo têm deser capazes de fazer tarefas de alto-nível como raciocinar, aprender e deduzir, entre outras. 
O desempenho de um jogador é ditado pela qualidade com que é capaz de executar as três seguintes tarefas: Interpretar as regras do jogo (representando de forma eficiente o conhecimento obtido), analizar as regras do jogo (reconhecendo padrões e outras características) e criando uma estratégia para esse jogo (fazendo uma pesquisa rápida pelas acções possíveis).

Neste trabalho é dada uma visão geral de GPP e o estado da arte da àrea é apresentado.


% Keywords
\begin{flushleft}

\palavrasChave{General Game Playing, Planeamento em Múltiplos Jogos, Inteligência Artificial}

\end{flushleft}

\end{abstract}
\end{otherlanguage}
