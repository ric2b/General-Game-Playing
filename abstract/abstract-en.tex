%!TEX root = ../dissertation.tex

\begin{otherlanguage}{english}
\begin{abstract}
% Set the page style to show the page number
\thispagestyle{plain}
\abstractEnglishPageNumber
Artificial Intelligence research has lead to many systems that perform incredibly well for the specific task they were designed for, but are useless for anything else.
For example, while Deep Blue could defeat Garry Kasparov, the world's chess champion, it was utterly incapable of playing checkers or even tic-tac-toe. Also, with these systems, all the interesting analysis of the game is done ahead of time and is done by the developers, not the system.
General Game Playing (GGP) aims to create computer systems capable of playing any game without human assistance, being given only the game rules. Such systems must be highly adaptable and should be capable of high-level tasks such as reasoning, learning and deduction, among others.
Player performance is dictated by how well it handles the following three tasks:
Interpreting the game rules (efficiently representing this knowledge), analyzing the game rules (recognizing patterns and other features) and deriving a strategy for the given game (quickly searching through the possible actions). 
%These tasks depend on three characteristics: How efficiently does the system represent it's knowledge, how deep it's analysis of the rules goes and how fast it can search through the possible moves for the game.

In this work an overview of GGP is given and the state of the art solutions are presented.


% Keywords
\begin{flushleft}

\keywords{}
General Game Playing,
Multi-Game Playing,
Artificial Intelligence,


\end{flushleft}

\end{abstract}
\end{otherlanguage}
